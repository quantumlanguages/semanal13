% tipo
\documentclass{article}

% formato
\usepackage[letterpaper, margin=1.5cm]{geometry}

% fuente
\renewcommand{\familydefault}{\sfdefault}

% idioma
\usepackage[spanish]{babel}

% lógica
\usepackage{ebproof}

% macros
\newcommand{\x}{\times}
\newcommand{\m}{\rightarrow}

% encabezado
\title{
    Lenguajes de Programación 2020-1\\
    Facultad de Ciencias UNAM\\
    Ejericio Semanal 13
}

\author{
    Sandra del Mar Soto Corderi \qquad
    Edgar Quiroz Castañeda
}

\date{
    Fecha de entrega: 21 de noviembre de 2019
}

% documento

\begin{document}
    \maketitle

    \begin{enumerate}
        \item Considera los siguientes aximas de subtipado:
        \[D \leq B \quad E \leq B \quad B \leq A \quad C \leq A\]

        \begin{enumerate}
            \item Muestre formalmente si el subtipado
            \[
                (E \x D \x A \m \{\ell:A\}) \m C \leq 
                (B \x A \m \{\ell':A, \ell:D\}) \m A
            \]
            es válido, de no serlo explica por qué.

            Suponiendo que es válido, tendríamos que su árbol de derivación 
            existe.

            \[
                \begin{prooftree}
                    \hypo{Error}
                    \infer 1 {E \x D \x A \leq  B \x A}

                    \hypo{}
                    \infer 1 [ampl]{\{\ell' : A, \ell : D\} \leq \{\ell : A\}}

                    \infer 2 [func]{
                        B \x A \m \{\ell' : A, \ell : D\} \leq 
                        E \x D \x A \m \{\ell : A\}
                    }

                    \hypo{Axioma}
                    \infer 1 {C \leq A}

                    \infer 2 [func] {
                        E \x D \x A \m \{\ell : A\} \m C \leq
                        B \x A \m \{\ell' : A, \ell : D\} \m A
                    }
                \end{prooftree}
            \]

            Como los subtipos de tipos productos están definidos entrada por
            entrada, al tener diferente cantidad de términos automáticamente no
            son subtipos.

            \item Agrega \textbf{una sóla} regla de subtipado que haga válidp el
            subtipado del inciso anterior y muestra su derivación formal con el 
            nuevo sistema.

            Se puede agregar una regla para currificar un tipo usando
            registros en lugar de productos.

            \[
                \begin{prooftree}
                    \hypo{T \x T_0 \m \{\ell_1 : T_1, \dots, \} \leq 
                    S \x S_0 \m \{\ell_1 : S_1, \dots, \}}
                    \infer 1 [Quasi-currying]
                    {T \m \{\ell_0 : T_0, \ell_1 : T_1, \dots, \} \leq 
                    S \x S_0 \m \{\ell_1 : S_1, \dots, \}}
                \end{prooftree}
            \]

            Por lo que la derivación completa quedaría como 
            \[
                \begin{prooftree}
                    \hypo{Axioma}
                    \infer 1 {E \leq B}

                    \infer 0 [trans] {D \leq A}

                    \infer 0 [ref] {A \leq A}

                    \infer 3 {E \x D \x A \leq B \x A \x A}

                    \hypo{Axioma}
                    \infer 1 {D \leq A}

                    \infer 1 [prof] {\{\ell : D\} \leq \{\ell : A\}}

                    \infer 2 [func] {
                        B \x A \x A \m \{\ell : D\} \leq 
                        E \x D \x A \m \{\ell : A\}
                    }

                    \infer 1 [curr] {
                        B \x A \m \{\ell' : A, \ell : D\} \leq 
                        E \x D \x A \m \{\ell : A\}
                    }

                    \hypo{Axioma}
                    \infer 1 {C \leq A}

                    \infer 2 [func] {
                        E \x D \x A \m \{\ell : A\} \m C \leq
                        B \x A \m \{\ell' : A, \ell : D\} \m A
                    }
                \end{prooftree}
            \]

        \end{enumerate}
    \end{enumerate}
\end{document}